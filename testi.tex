\documentclass[12pt,a4paper]{article}
\usepackage[utf8]{inputenc}
\usepackage[finnish]{babel}
\usepackage[T1]{fontenc}
\usepackage{amsmath}
\usepackage{amsfonts}
\usepackage{amssymb}
\usepackage{amsthm}
\newcommand{\R}{\mathbb{R}}
\newcommand{\N}{\mathbb{N}}
\pagestyle{plain}
\title{Kandi}
\author{Joni Hanski}
\date{20.4.2016}
\begin{document}
\maketitle
Kandissamme haluamme osoittaa metrisissä avaruuksissa sulkeuman karakterisoinnin jonon raja-arvojen avulla yleistyvän yleisiin topologisiin avaruuksiin käyttämällä jonojen sijaan konstruktiota \textbf{suunnattu verkko}. Kandi kuuluu topologian alaan.

Osoitamme aluksi miten metrisessä avaruudessa sulkeuma voidaan karakterisoida jonojen raja-arvojen avulla, ja tämän jälkeen tutkimme ei-metrisen avaruuden tapausta jossa sama tulos ei enää päde. Tämän jälkeen johdamme suunnattuja verkkoja käyttävän tuloksen yleisiin topologisiin avaruuksiin, ja näytämme kuinka tämä määrittely johtaa eri lopputulokseen esimerkkitapauksessamme.


\end{document}