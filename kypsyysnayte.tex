\documentclass[12pt,a4paper,leqno,notitlepage]{report}

\usepackage[utf8]{inputenc}
\usepackage[T1]{fontenc}
\usepackage[finnish]{babel}
\usepackage{amsthm}
\usepackage{amsfonts}         
\usepackage{amsmath}
\usepackage{amssymb}

\pagestyle{plain}
\setcounter{page}{1}
\addtolength{\hoffset}{-1.15cm}
\addtolength{\textwidth}{2.3cm}
\addtolength{\voffset}{0.45cm}
\addtolength{\textheight}{-0.9cm}

\title{Tiivistelmä}

\author{Joni Hanski}

\begin{document}
\maketitle

Kandidaatintutkielmassamme haluamme osoittaa metrisissä avaruuksissa sulkeuman karakterisoinnin jonon raja-arvojen avulla yleistyvän yleisiin topologisiin avaruuksiin käyttämällä jonojen sijaan yleisempää rakennetta verkko. Tutkielma kuuluu topologian alaan, ja käsittelee pääosin kurssin Topologia II aiheita.

Luvut 2-4 kertaavat kurssien Topologia I ja Topologia II keskeisimpiä tuloksia. Näistä luku 2 kertaa yleisiä topologisia avaruuksia koskevia käsitteitä. Luku 3 sisältää erityisesti kurssin Topologia I kannalta keskeisen metrisen avaruuden määritelmän. Metrinen avaruus on erityistapaus yleisestä topologisesta avaruudesta. Luvun 4 keskeinen tulos on osoittaa jonojen suppenemisen toimivan sulkeuman määrittämiseen metrisessä avaruudessa.

Luvussa 5 osoitamme että luvun 4 tulos ei kuitenkaan päde kaikilla topologisilla avaruuksilla. Osoitamme tämän tutkimalla erään yksittäisen ei-metrisen topologisen avaruuden ominaisuuksia.

Luvussa 6 annamme tutkielman päätuloksen. Määrittelemme ensin verkon käsitteen yleistyksenä jonon käsitteelle ja, vahvasti luvun 4 tulosta mukaillen, osoitamme verkkojen suppenemisen toimivan sulkeuman määrittämiseen kaikissa topologisissa avaruuksissa.
\end{document}
