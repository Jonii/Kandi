\documentclass[12pt,a4paper,leqno]{report}

\usepackage[utf8]{inputenc}
\usepackage[T1]{fontenc}
\usepackage[finnish]{babel}
\usepackage{amsthm}
\usepackage{amsfonts}         
\usepackage{amsmath}
\usepackage{amssymb}

\newcommand{\R}{\mathbb{R}}
\newcommand{\C}{\mathbb{C}}
\newcommand{\Q}{\mathbb{Q}}
\newcommand{\N}{\mathbb{N}}
\newcommand{\No}{\mathbb{N}_0}
\newcommand{\Z}{\mathbb{Z}}
\newcommand{\X}{\mathrm{X}}
\newcommand{\Pow}{\mathbb{P}}
\newcommand{\B}{\mathcal{B}}
\newcommand{\T}{\mathcal{T}}
\newcommand{\I}{\mathrm{I}}
\newcommand{\diam}{\operatorname{diam}}

\theoremstyle{plain}
\newtheorem{lause}[equation]{Lause}
\newtheorem{lem}[equation]{Lemma}
\newtheorem{prop}[equation]{Propositio}
\newtheorem{kor}[equation]{Korollaari}

\theoremstyle{definition}
\newtheorem{maar}[equation]{Määritelmä}
\newtheorem{konj}[equation]{Konjektuuri}
\newtheorem{esim}[equation]{Esimerkki}

\theoremstyle{remark}
\newtheorem{huom}[equation]{Huomautus}

\pagestyle{plain}
\setcounter{page}{1}
\addtolength{\hoffset}{-1.15cm}
\addtolength{\textwidth}{2.3cm}
\addtolength{\voffset}{0.45cm}
\addtolength{\textheight}{-0.9cm}

\title{Kandi}
\author{Joni Hanski}
\date{20.4.2016}

\begin{document}

\maketitle

\tableofcontents

\chapter{Johdanto}\label{johd}

Kandissamme haluamme osoittaa metrisissä avaruuksissa sulkeuman karakterisoinnin jonon raja-arvojen avulla yleistyvän yleisiin topologisiin avaruuksiin käyttämällä jonojen sijaan konstruktiota \textbf{suunnattu verkko}. Kandi kuuluu topologian alaan, ja suurimmaksi osaksi sisältyy kurssiin Topologia II.

Osoitamme aluksi miten metrisessä avaruudessa sulkeuma voidaan karakterisoida jonojen raja-arvojen avulla, ja tämän jälkeen tutkimme ei-metrisen avaruuden tapausta jossa sama tulos ei enää päde. Tämän jälkeen johdamme suunnattuja verkkoja käyttävän tuloksen yleisiin topologisiin avaruuksiin, ja näytämme kuinka tämä määrittely johtaa eri lopputulokseen esimerkkitapauksessamme.

\chapter{Topologia}\label{TOP}

Topologian avulla voidaan puhua avaruuden rakenteesta. Tietyllä tapaa topologia kuvaa sitä, kuinka lähellä tai kaukana pisteet ovat toisistaan.

\begin{maar}\label{topmaar}
\emph{Topologia}. Olkoon $\X$ avaruus ja $\T \subset \Pow(\X)$ joukon $\X$ osajoukkojen joukko. Sanomme, että $\T$ on joukon $\X$ \emph{topologia}, mikäli seuraavat ehdot pätevät:

(T1) $\T$ sisältää osajoukkojensa mielivaltaiset yhdisteet

(T2) Jos $A,B \in \T$, niin $A \cap B \in \T$

(T3) $\X, \varnothing \in \T$
\end{maar}

Jos joukolle $\X$ on määrätty topologia $\T$, sanomme että $\X$ on \emph{topologinen avaruus}.

Jos joukko $A \subset \X$ kuuluu avaruuden $\X$ topologiaan $\T$, sanomme että joukko $A$ on $\T$-\emph{avoin}, tai lyhyesti \emph{avoin}, mikäli topologia on asiayhteydestä selvä. Mikäli joukon $A \subset \X$ komplementti $\X \setminus A$ on avoin, sanomme että A on \emph{suljettu}. Joukko voi olla avoin, suljettu, molemmat tai ei kumpikaan. Esimerkiksi koko avaruus $\X$ on aina sekä avoin että suljettu.

Oletamme tästedes, että $\X$ on topologinen avaruus.

\begin{maar}\label{ymp}
\emph{Ympäristö}. Olkoon $x \in \X$ ja $A \subset \X$. Sanomme että $A$ on pisteen $x$ \emph{ympäristö}, mikäli $A$ on avoin ja $x \in A$.
\end{maar}

\begin{maar}\label{sulk}
\emph{Sulkeuma}. Joukon $A \subset \X$ \emph{sulkeuma} määritellään niiden pisteiden joukkona, joiden jokainen ympäristö leikkaa joukon $A$.
\begin{equation}
\overline{A} = \text{sulkeuma }A = \{ x \in \X \mid \text{jos } x \in U \in \T \text{ pätee } U \cap A \not= \varnothing \}
\end{equation}
\end{maar}

Joukon $A$ sulkeuma on suljettu, ja sisältyy jokaiseen suljettuun joukkoon joka sisältää joukon $A$. Intuitiivisesti tämä seuraa siitä, että sulkeuman komplementti on yhdiste kaikista avoimista joukoista jotka eivät leikkaa joukkoa $A$.

\begin{maar}\label{kantamaar}
\emph{Kanta}. Käytännössä voi olla hankala listata jokainen tietyn topologian avoin joukko määritellessä topologiaa. Tämän sijaan usein puhutaan rajatummasta joukosta avoimia joukkoja, jotka yksikäsitteisesti määräävät topologian. Joukko $\B \subset \Pow(\X)$ on topologian \emph{kanta} joukossa $\X$, mikäli seuraavat ehdot pätevät:

(K0) $\varnothing \in \B$.

(K1) $\bigcup_{B \in \B} B = \X$.

(K2) Jos $x \in B_1 \cap B_2$ ja $B_1,B_2\in \B$, tällöin on olemassa $B \in \B$ siten, että $x \in B \subset B_1 \cap B_2$.
\end{maar}

Joukko $U \subset \X$ kuuluu kannan $\B$ määräämään topologiaan jos ja vain jos $U$ on jokin yhdiste kannan alkioista.

\chapter{Metrinen avaruus}\label{MET}

Metrisen avaruuden keskeinen käsite on metriikka, joka pyrkii formalisoimaan etäisyyden käsitteen. Kandin aiheen kannalta erityisen huomionarvoista on se, että metriikka määrää yksiselitteisesti avaruudelle luonnollisen topologian.

\begin{maar}\label{metmaar}egation.
De Morgans Law of Set Theory Proof - Math Theorems
Statement:
 
\emph{Metriikka}. Joukossa $\X$ määritelty funktio $d : \X \times \X \to \R$ on \emph{metriikka} joukossa $X$ jos pätee

(M1) Kaikilla $a, b \in \X$ pätee $d(a,b) \geq 0$.

(M2) $d(a,b) = 0$ jos ja vain jos $a = b$.

(M3) (Kolmioepäyhtälö) Kaikilla $a, b, c \in \X$ pätee $d(a,c) \leq d(a,b) + d(b,c)$.
\end{maar}

\begin{maar}\label{pallomar}
\emph{Palloympäristö}. Pisteen $x \in \X$ $r$-säteinen \emph{palloympäristö} on joukko $B(x,r) = \{ y \in \X \mid d(x, y) < r \}$.
\end{maar}

Metriselle avaruudelle on hyvin luonnollista määrätä topologia käyttämällä kantana avaruuden eri palloympäristöjä. Jos siis $\X$ on metrinen avaruus, sen kannaksi valitaan $\B = \{ B(x,r) \mid x \in \X, r \in \R \}$. Kutsumme kannan $\B$ määräämää topologiaa nyt \emph{metriikan indusoimaksi topologiaksi} joukossa $\X$.

\chapter{Sulkeuman karakterisointi jonojen avulla}

Palautetaan ensin mieleen jonon määritelmä. Sanomme, että kuvaus $x : \N \to \X$ on jono avaruudessa $\X$, ja merkitsemme $x_n := x(n)$. Erityisesti jonon suppeneminen ja raja-arvo on työmme kannalta keskeistä.

\begin{maar}\label{jonosupmaar}
\emph{Jonon suppeneminen}. Olkoon $\X$ topologinen avaruus, ja olkoon $a$ jono avaruudessa $\X$. Sanomme, että jono \emph{suppenee} kohti pistettä $x$, mikäli jokaista pisteen $x$ ympäristöä $U \subset \X$ kohti voidaan valita jonon $a$ indeksi $n_U$ siten, että kaikilla $n > n_U$ pätee $a_n \in U$. Tiettyjen ehtojen pätiessä avaruuden jono voi supeta vain yhtä pistettä kohti. Tällöin jos jono suppenee kohti pistettä $x$, voimme sanoa että $x$ on jonon \emph{raja-arvo}.
\end{maar}

Todistamatta mainitsemme että metrisessä avaruudessa jono voi supeta vain yhtä pistettä kohti.

Metrisessä avaruudessa voimme nyt osoittaa, että joukon $A$ jonojen raja-arvojen joukko on sama kuin joukon $A$ sulkeuma. Tämä tarkoittaa, että piste $x$ kuuluu joukon $A$ sulkeumaan jos ja vain jos on sellainen jono $a : \N \to A$ että tämän raja-arvo on $x$. Tämän osoittamiseksi... TODO

\chapter{Erikoinen ei-metrinen avaruus}
Aiemmin huomasimme että metrisessä avaruudessa jonon raja-arvojen avulla voidaan määrittää sulkeuma. Nyt tutkimme ei-metristä avaruutta jossa tämä karakterisointi ei päde.

Olkoon $\R$ avaruus, jossa topologian $\T$ määrittelee seuraava ehto:

\begin{equation}
\T = \{U \subset \R \mid \R \setminus U \text{ on numeroituva} \} \cup \{ \varnothing \}
\end{equation}

Osoitamme ensin että $\T$ määrää topologian.

Ehdot 1 ja 2 voidaan osoittaa De Morganin laeilla, eli 

\begin{align}
(A \cap B)^C &= A^C \cup B^C \\
(A \cup B)^C &= A^C \cap B^C
\end{align}

Ehdon 1 kohdalla huomaamme että jos joukon $A$ komplementti on numeroituva, tällöin yhdisteen $A \cup B$ komplementti sisältyy joukon $A$ komplementtiin. Toisaalta ehto 2 toteutuu kun huomaamme että kahden numeroituvan joukon yhdiste on yhä numeroituva, eli kahden joukon leikkauksen komplementti on yhä numeroituva jos molempien alkuperäisten joukkojen komplementti oli numeroituva.

Ehto 3 toteutuu selvästi.

Tutkitaan erityisesti yksikköväliä $\I = [0,1]$. Joukon $\I$ sulkeuma osoittautuu olevan koko avaruus $\X$. Tämän osoittamiseksi teemme vastaoletuksen, eli oletamme että löytyy $x \in \X$ joka ei kuulu joukon $\I$ sulkeumaan. Tällöin pisteellä $x$ on ainakin yksi ympäristö $U \subset \X$ joka ei leikkaa joukkoa $\I$. Tällöin $\I \subset U^C$. Koska tästä seuraisi että $U^C$ on ylinumeroituva, tämä on ristiriita, eli vastaoletuksemme oli väärä.

Toisaalta voimme osoittaa että jos $x \in \R$ ja $x \not\in \I$, mikään joukon $\I$ jono ei suppene kohti pistettä $x$. Teemme jälleen vastaoletuksen, eli että löytyy jono $a$ joukossa $\I$ joka suppenee kohti pistettä $x$.

Nyt voimme valita pisteelle $x$ ympäristön $U \subset \R$ siten, että
\begin{equation}
U = \R \setminus \{x \in \R \mid a_i = x \text{ jollain } i \in \N \}
\end{equation}

Selvästi tämä joukko on avoin, koska tämän sulkeuma on jonon arvojoukko. Selvästi joukko $U$ myös sisältää pisteen $x$, sillä jonon arvot jäävät joukkoon $\I$ mutta $x \not\in \I$. Täten $U$ on pisteen $x$ ympäristö, ja toisaalta myös selvästi jono ei ikinä saa arvoja tässä joukossa. 

\begin{thebibliography}{9}

\bibitem{Topo}
Jussi Väisälä: Topologia II
\end{thebibliography}

\end{document}
