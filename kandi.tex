\documentclass[12pt,a4paper,leqno]{report}

\usepackage[utf8]{inputenc}
\usepackage[T1]{fontenc}
\usepackage[finnish]{babel}
\usepackage{amsthm}
\usepackage{amsfonts}         
\usepackage{amsmath}
\usepackage{amssymb}

\newcommand{\R}{\mathbb{R}}
\newcommand{\C}{\mathbb{C}}
\newcommand{\Q}{\mathbb{Q}}
\newcommand{\N}{\mathbb{N}}
\newcommand{\No}{\mathbb{N}_0}
\newcommand{\Z}{\mathbb{Z}}
\newcommand{\X}{\mathrm{X}}
\newcommand{\Pow}{\mathbb{P}}
\newcommand{\B}{\mathcal{B}}
\newcommand{\T}{\mathcal{T}}
\newcommand{\I}{\mathrm{I}}
\newcommand{\D}{\mathcal{D}}
\newcommand{\diam}{\operatorname{diam}}
\newcommand{\verkop}{\leq}

\theoremstyle{plain}
\newtheorem{lause}[equation]{Lause}
\newtheorem{lem}[equation]{Lemma}
\newtheorem{prop}[equation]{Propositio}
\newtheorem{kor}[equation]{Korollaari}

\theoremstyle{definition}
\newtheorem{maar}[equation]{Määritelmä}
\newtheorem{konj}[equation]{Konjektuuri}
\newtheorem{esim}[equation]{Esimerkki}

\theoremstyle{remark}
\newtheorem{huom}[equation]{Huomautus}

\pagestyle{plain}
\setcounter{page}{1}
\addtolength{\hoffset}{-1.15cm}
\addtolength{\textwidth}{2.3cm}
\addtolength{\voffset}{0.45cm}
\addtolength{\textheight}{-0.9cm}

\title{Kandi}
\author{Joni Hanski}
\date{2.6.2016}

\begin{document}

\maketitle

\tableofcontents

\chapter{Johdanto}\label{johd}

Kandidaatintutkielmassamme haluamme osoittaa metrisissä avaruuksissa sulkeuman karakterisoinnin jonon raja-arvojen avulla yleistyvän yleisiin topologisiin avaruuksiin käyttämällä jonojen sijaan konstruktiota \textbf{suunnattu verkko}. Tutkielma kuuluu topologian alaan, ja suurimmaksi osaksi sisältyy kurssiin Topologia II.

Osoitamme aluksi miten metrisessä avaruudessa sulkeuma voidaan karakterisoida jonojen raja-arvojen avulla, ja tämän jälkeen tutkimme ei-metrisen avaruuden tapausta jossa sama tulos ei enää päde. Tämän jälkeen johdamme suunnattuja verkkoja käyttävän tuloksen yleisiin topologisiin avaruuksiin.

\chapter{Topologia}\label{TOP}

Topologian avulla voidaan puhua avaruuden rakenteesta. Tietyllä tapaa topologia kuvaa sitä, kuinka lähellä tai kaukana pisteet tai joukot ovat toisistaan.

\begin{maar}\label{topmaar}
\emph{Topologia}. Olkoon $\X$ avaruus ja $\T \subset \Pow(\X)$ joukon $\X$ osajoukkojen joukko. Sanomme, että $\T$ on joukon $\X$ \emph{topologia}, mikäli seuraavat ehdot pätevät:

(T1) $\T$ sisältää alkioidensa mielivaltaiset yhdisteet,

(T2) Jos $A,B \in \T$, niin $A \cap B \in \T$,

(T3) $\X, \varnothing \in \T$.
\end{maar}

Jos joukolle $\X$ on määrätty topologia $\T$, sanomme että $\X$ on \emph{topologinen avaruus}.

Jos joukko $A \subset \X$ kuuluu avaruuden $\X$ topologiaan $\T$, sanomme että joukko $A$ on $\T$-\emph{avoin}, tai lyhyesti \emph{avoin}, mikäli topologia on asiayhteydestä selvä. Mikäli joukon $A \subset \X$ komplementti $\X \setminus A$ on avoin, sanomme että A on \emph{suljettu}. Joukko voi olla avoin, suljettu, molemmat tai ei kumpikaan. Esimerkiksi koko avaruus $\X$ on aina sekä avoin että suljettu.

Oletamme tästedes, että $\X$ on topologinen avaruus.

\begin{maar}\label{ymp}
\emph{Ympäristö}. Olkoon $x \in \X$ ja $A \subset \X$. Sanomme että $A$ on pisteen $x$ \emph{ympäristö}, mikäli $A$ on avoin ja $x \in A$.
\end{maar}

\begin{maar}\label{sulk}
\emph{Sulkeuma}. Joukon $A \subset \X$ \emph{sulkeuma}, merkitään $\overline{A}$, määritellään niiden pisteiden joukkona, joiden jokainen ympäristö leikkaa joukon $A$.
\begin{equation}
\overline{A} = \{ x \in \X \mid \text{jos } U \text{ on pisteen } x \text{ ympäristö, niin pätee } U \cap A \not= \varnothing \}\text{.}
\end{equation}
\end{maar}

Joukon $A$ sulkeuma on suljettu, ja sisältyy jokaiseen suljettuun joukkoon joka sisältää joukon $A$. Intuitiivisesti tämä seuraa siitä, että sulkeuman komplementti on yhdiste kaikista avoimista joukoista jotka eivät leikkaa joukkoa $A$.

\begin{maar}\label{kantamaar}
\emph{Kanta}. Käytännössä voi olla hankala listata jokainen tietyn topologian avoin joukko määritellessä topologiaa. Tämän sijaan usein puhutaan rajatummasta joukosta avoimia joukkoja, jotka yksikäsitteisesti määräävät topologian. Olkoon joukko $\X$ avaruus jonka topologian määrittää topologia $\T$. Joukko $\B \subset \Pow(\X)$ on topologian $\T$ \emph{kanta} joukossa $\X$, mikäli seuraavat ehdot pätevät:

(1) $\B \subset \T$,

(2) Jokainen $U \in \T, U \not= \varnothing$ voidaan lausua yhdisteenä joistain kannan $\B$ jäsenistä.
\end{maar}

Toisin sanoen, epätyhjä joukko $U \subset \X$ on avoin topologiassa $\T$ jos ja vain jos $U$ on yhdiste kannan alkioista. Kanta ei ole yksikäsitteinen, samalle topologialle voidaan löytää useita kantoja.

\begin{lause}\label{kantalause}
\emph{Kantalause. Topologia II \cite{Topo} Lause 2.9}. Olkoon $\X$ joukko, ja $\B \subset \Pow(\X)$. $\B$ on tällöin jonkin topologian kanta, jos ja vain jos seuraavat ehdot pätevät:

(K1) $\B$ on joukon $\X$ peite,

(K2) Jos $B_1, B_2 \in \B$, ja jos $x \in B_1 \cap B_2$, niin on olemassa $B \in \B$ siten, että $x \in B \subset B_1 \cap B_2$.
\end{lause}

\noindent\emph{Todistus}. Sivuutamme todistuksen. Lause on todistettu kirjassa Topologia II \cite{Topo}.

\chapter{Metrinen avaruus}\label{MET}

Metrisen avaruuden keskeinen käsite on metriikka, joka pyrkii formalisoimaan etäisyyden käsitteen. Kandidaatintutkielman aiheen kannalta erityisen huomionarvoista on se, että metriikka indusoi avaruudelle topologian, ja metriikan indusoimilla topologioilla on monia hyödyllisiä erityisominaisuuksia.

\begin{maar}\label{metmaar}
\emph{Metriikka}. Joukossa $\X$ määritelty funktio $d : \X \times \X \to \R$ on \emph{metriikka} joukossa $X$, jos pätee

(M1) Kaikilla $a, b \in \X$ pätee $d(a,b) \geq 0$,

(M2) Kaikilla $a, b \in \X$ pätee $d(a,b) = d(b,a)$,

(M3) $d(a,b) = 0$ jos ja vain jos $a = b$,

(M4) (Kolmioepäyhtälö) Kaikilla $a, b, c \in \X$ pätee $d(a,c) \leq d(a,b) + d(b,c)$.
\end{maar}

\begin{maar}\label{pallomar}
\emph{Palloympäristö}. Pisteen $x \in \X$ $r$-säteinen \emph{palloympäristö} on joukko $B(x,r) = \{ y \in \X \mid d(x, y) < r \}$.
\end{maar}

Metriselle avaruudelle on hyvin luonnollista määrätä topologia käyttämällä kantana avaruuden eri palloympäristöjä. Jos siis $\X$ on metrinen avaruus, voimme valita sen kannaksi joukon $\B = \{ B(x,r) \mid x \in \X, r \in \R_{> 0} \}$. Kutsumme kannan $\B$ määräämää topologiaa nyt \emph{metriikan indusoimaksi topologiaksi} joukossa $\X$.

\chapter{Sulkeuman karakterisointi jonojen avulla}

Palautetaan ensin mieleen jonon määritelmä. Sanomme, että kuvaus $x : \N \to \X$ on jono avaruudessa $\X$, ja merkitsemme $x_n := x(n)$. Erityisesti jonon suppeneminen ja raja-arvo on työmme kannalta keskeistä.

\begin{maar}\label{jonosupmaar}
\emph{Jonon suppeneminen}. Olkoon $\X$ topologinen avaruus, ja olkoon $a$ jono avaruudessa $\X$. Sanomme, että jono $a$ \emph{suppenee} kohti pistettä $x$, mikäli jokaista pisteen $x$ ympäristöä $U \subset \X$ kohti voidaan valita jonon $a$ indeksi $n_U$ siten, että kaikilla $n > n_U$ pätee $a_n \in U$. Tiettyjen ehtojen pätiessä avaruuden jono voi supeta vain yhtä pistettä kohti. Tällöin jos jono suppenee kohti pistettä $x$, voimme sanoa että $x$ on jonon \emph{raja-arvo}.
\end{maar}

Todistamatta mainitsemme että metrisessä avaruudessa jono voi supeta vain yhtä pistettä kohti. Tämä on todistettu esimerkiksi kirjassa Topologia I \cite{Topo1} lause 11.4.

\begin{lause}
Olkoon $\X$ metrinen avaruus, ja $A \subset \X$ joukko. Tällöin pätee
\begin{equation}
\overline{A} = \{x \in \X \mid \text{Löytyy jono } n : N \to A \text{ joka suppenee kohti pistettä } x\}\text{.}
\end{equation}
Toisin sanoen, joukon jonojen raja-arvojen joukko on sama, kuin joukon sulkeuma.
\end{lause}

\noindent\emph{Todistus}.

\emph{Kohta 1}. Osoitetaan, että jos jono $a : \N \to A$ suppenee kohti pistettä $x \in \X$, piste $x$ kuuluu joukon $A$ sulkeumaan.

Sulkeuman määritelmän perusteella väite on yhtäpitävä sen kanssa, että osoitamme jokaisen pisteen $x$ ympäristölle $U \subset \X$ pätevän, että $U \cap A \not= \varnothing$.

Olkoon $U$ jokin pisteen $x$ ympäristö. Koska jono $a$ suppenee kohti pistettä $x$, jostain indeksistä $n_U$ lähtien jonon arvot sisältyvät joukkoon $U$, ja erityisesti pätee $a_{n_U} \in U$. Koska $a_{n_U} \in A$, tällöin $a_{n_U} \in A \cap U$, eli $A \cap U \not= \varnothing$.

\emph{Kohta 2}. Osoitetaan, että jos piste $x$ kuuluu joukon $A$ sulkeumaan, löytyy jono $a : \N \to A$ joka suppenee kohti pistettä $x$.

Valitaan jono $a$ siten, että $a_n \in A \cap B(x, 2^-n)$, eli jokainen alkio kuuluu $2^-n$-säteisen pisteen $x$ palloympäristön, ja joukon $A$ leikkaukseen. Koska jokainen edellä kuvatunlainen palloympäristö kuuluu topologian kantaan ja on täten avoin, ja toisaalta koska piste $x$ kuuluu joukon $A$ sulkeumaan, on jokainen leikkaus epätyhjä, eli valinta voidaan tehdä.

Olkoon joukko $U \subset \X$ jokin pisteen $x$ ympäristö. Koska topologian $\T$ kanta on palloympäristöjen joukko, voidaan joukko $U$ lausua näiden kannan alkioiden yhdisteenä, ja siis löytyy jokin kannan alkio, palloympäristö $B(y,r)$ joka sisältää pisteen $x$ ja sisältyy joukkoon $U$, eli $x \in B \subset U$.

Koska $d(x,y) < r$, löytyy luku $\epsilon \in \R$ jolle pätee $d(x,y) + \epsilon < r$, ja täten voimme kolmioepäyhtälön avulla osoittaa että palloympäristö $B^* := B(x,\epsilon)$ sisältyy palloympäristöön $B(y,r)$. Jos piste $z$ sisältyy palloympäristöön $B(x,\epsilon)$, tällöin kolmioepäyhtälön avulla saamme

\begin{equation}
d(y,z) < d(y,x) + d(x,z) < d(y,x) + \epsilon < r\text{.}
\end{equation}

eli piste $z$ sisältyy palloympäristöön $B(y,r)$. Edeltävässä yhtälössä ensimmäisessä epäyhtälössä käytimme kolmioepäyhtälöä, toisessa tietoa siitä, että $z \in B(x,\epsilon)$.

Toisaalta, koska joukon palloympäristön $B^* = \{z \in \X \mid d(z,y) < r\}$ säde $r$ on positiivinen reaaliluku, voidaan valita tämän ja pisteen $0$ väliltä jokin muotoa $2^-n$ oleva luku. Tällöin $a_n$ ja jokainen tätä seuraava jonon alkio sisältyy joukkoon $B$, eli jos $m > n$, pätee $a_m \in B \subset U$. Koska joukko $U$ oli valittu mielivaltaisesti, olemme nyt osoittaneet että jono $a$ saavuttaa ennenpitkää jokaisen pisteen $x$ ympäristön.

\chapter{Erikoinen ei-metrinen avaruus}
Aiemmin huomasimme että metrisessä avaruudessa jonon raja-arvojen avulla voidaan määrittää sulkeuma. Nyt tutkimme ei-metristä avaruutta jossa tämä karakterisointi ei päde.

Olkoon $\R$ avaruus, jossa topologian $\T$ määrittelee seuraava ehto:

\begin{equation}
\T = \{U \subset \R \mid \R \setminus U \text{ on numeroituva} \} \cup \{ \varnothing \}\text{.}
\end{equation}

Osoitamme ensin että $\T$ määrää topologian.

Ehdot T1 ja T2 voidaan osoittaa De Morganin laeilla, eli 

\begin{align}
(A \cap B)^C &= A^C \cup B^C \\
(A \cup B)^C &= A^C \cap B^C
\end{align}

Ehdon T1 kohdalla huomaamme että jos joukon $A$ komplementti on numeroituva, tällöin yhdisteen $A \cup B$ komplementti sisältyy joukon $A$ komplementtiin, eli $(A \cup B)^C$ on numeroituva. Mielivaltaisen yhdisteen tapauksessa $\bigcup_{i \in \mathbb{I}} A_i$ voimme valita yhden indeksin $j \in \mathbb{I}$ jolloin $\bigcup_{i \in \mathbb{I}} A_i = A_j \cup\bigcup_{i \in \mathbb{I}} A_i$, eli edeltävän perusteella myös mielivaltaisen yhdisteen komplementti on avoin. Toisaalta ehto T2 toteutuu kun huomaamme että kahden numeroituvan joukon yhdiste on yhä numeroituva, eli kahden joukon leikkauksen komplementti on yhä numeroituva jos molempien alkuperäisten joukkojen komplementti oli numeroituva.

Ehto T3 toteutuu selvästi.

Tutkitaan erityisesti yksikköväliä $\I = [0,1]$. Joukon $\I$ sulkeuma osoittautuu olevan koko avaruus $\R$. Tämän osoittamiseksi teemme vastaoletuksen, eli oletamme että löytyy $x \in \R$ joka ei kuulu joukon $\I$ sulkeumaan. Tällöin pisteellä $x$ on ainakin yksi ympäristö $U \subset \R$ joka ei leikkaa joukkoa $\I$. Tällöin $\I \subset U^C$. Koska tästä seuraisi että $U^C$ on ylinumeroituva, tämä on ristiriita joukon $U$ avoimuuden kanssa, eli vastaoletuksemme oli väärä.

Toisaalta voimme osoittaa että jos $x \in \R$ ja $x \not\in \I$, mikään joukon $\I$ jono ei suppene kohti pistettä $x$. Teemme jälleen vastaoletuksen, eli että löytyy jono $a$ joukossa $\I$ joka suppenee kohti pistettä $x$.

Nyt voimme valita pisteelle $x$ ympäristön $U \subset \R$ siten, että
\begin{equation}
U = \R \setminus \{x \in \R \mid a_i = x \text{ jollain } i \in \N \}\text{.}
\end{equation}

Selvästi tämä joukko on avoin, koska tämän joukon komplementti on jonon arvojoukko, joka on selvästi numeroituva. Selvästi joukko $U$ myös sisältää pisteen $x$, sillä jonon arvot jäävät joukkoon $\I$ mutta $x \not\in \I$. Täten $U$ on pisteen $x$ ympäristö, ja toisaalta myös selvästi jonon alkiot eivät ikinä sisälly tähän joukkoon. Koska jonon suppeneminen vaatii että jokaiselle pisteen $x$ ympäristölle pätee että ennen pitkään jono saa arvonsa tässä joukossa, saamme ristiriidan, eli mikään yksikkövälin $\I$ jono ei suppene kohti pistettä $x$.

Täten sulkeuma on merkillisessä avaruudessamme laajempi kuin jonojen suppenemispisteet. Jotta saisimme kuvailtuja sulkeumaa suppenemisen kautta, meidän on käytettävä jonoja yleisempää käsitettä.

\chapter{Verkko}

Suunnattu verkko yleistää jonon käsitettä. Isolta osin verkon määritelmä vastaa jonon määritelmää, joten mainitsemme erikseen mikäli verkon ominaisuus poikkeaa jonosta.

Jono on kuvaus luonnollisilta luvuilta $\N$ kohdeavaruuteen $\X$. Verkko sen sijaan ottaa lähtöavaruudekseen jonkin \emph{suunnatun joukon}. Suunnattu joukko on määritelty seuraavasti:

\begin{maar}\label{suunjoukmaar}
\emph{Suunnattu joukko}. Olkoon joukko $\X$ jokin joukko, jossa on määritelty relaatio $\verkop$ joka toteuttaa seuraavat ehdot:

(V1) Jos $a, b, c \in \X$, ja $a \verkop b \verkop c$, tällöin $a \verkop c$ (transitiivisuus),

(V2) Jos $a \in \X$, tällöin $a \verkop a$,

(V3) Jos $a, b \in \X$, tällöin on olemassa $c \in \X$ siten, että $a \verkop c$ ja $b \verkop c$.

Kutsumme paria $(\X, \verkop)$ \emph{suunnatuksi joukoksi}.
\end{maar}

Ehto V3 erottaa tämän jonon määritelmästä. Toisin kuin luonnollisilla luvuilla, järjestyksemme on vain osittainen. Jos $a, b \in \X$, ei välttämättä päde $a \verkop b$ eikä $b \verkop a$. Ehto V3 kuitenkin antaa meille keinon puhua suppenemisesta liki identtisesti samalla tavoin kuin jonoilla.

\begin{maar}\label{verkmaar}
\emph{Verkko}. Olkoon $\D$ suunnattu joukko, $\X$ avaruus, ja $n : \D \to \X$ kuvaus suunnatusta joukosta $\D$ avaruuteen $\X$. Tällöin sanomme että $n$ on avaruuden $\X$ \emph{verkko}. Mikäli kaikki verkon $n$ arvot kuuluvat avaruuden $\X$ osajoukkoon $A$, sanomme että $n$ on verkko joukossa $A$.
\end{maar}

Osoittautuu että aiemmin käyttämämme määritelmä jonon suppenemiselle kelpaa lähes sellaisenaan verkon suppenemisen määrittelemiselle, korvaamalla jonon verkolla.

\begin{maar}\label{verksupmaar}
\emph{Verkon suppeneminen}. Olkoon $\X$ topologinen avaruus, ja $n$ avaruuden $\X$ verkko. Sanomme, että verkko \emph{suppenee} kohti pistettä $x$ mikäli jokaiselle pisteen $x$ ympäristölle $U$ löytyy joukon $\D$ alkio $d_U \in \D$ siten, että jos $d_U \verkop d$, tällöin $n(d) \in U$.
\end{maar}

Nyt voimme osoittaa että mielivaltaisessa topologisessa avaruudessa, piste kuuluu joukon sulkeumaan jos ja vain jos jokin tämän joukon verkko suppenee kohti tätä pistettä.

\begin{lause}\label{verksulklause}
Olkoon $\X$ avaruus, ja $A \subset \X$ tämän avaruuden osajoukko. Jos piste $x \in \X$ kuuluu joukon $A$ sulkeumaan, löytyy verkko $n : \D \to A$ joka suppenee kohti pistettä $x$, ja kääntäen, mikäli on olemassa verkko $n : \D \to A$ joka suppenee kohti pistettä $x$, tällöin $x \in \overline{A}$.
\end{lause}

\noindent\emph{Todistus}.

\emph{Suunta $\Leftarrow$}

Olkoon $\X$ avaruus, $A \subset \X$ joukko ja $n : \D \to A$ verkko joka suppenee kohti pistettä $x \in \X$. Osoitamme että piste $x$ kuuluu joukon $A$ sulkeumaan.

Olkoon $U \subset \X$ jokin pisteen $x$ ympäristö. Koska verkko $n$ suppenee kohti pistettä $x$, löytyy verkon lähtöavaruuden alkio $d_U$ jolle pätee, mikäli $d_U \verkop d$, tällöin $n(d) \in U$. Täten siis $n(d_U) \in U$, mutta toisaalta myös verkon määritelmän perusteella $n(d_U) \in A$, eli $n(d_U) \in U \cap A \not= \varnothing$.

\emph{Suunta $\Rightarrow$}

Olkoon $\X$ avaruus, $A \subset \X$ joukko ja $x \in \overline{A}$ joukon $A$ sulkeuman piste. Osoitamme että löytyy verkko $n : \D \to A$ joka suppenee kohti pistettä $x$.

Määritellään joukko $\D$ seuraavasti:

\begin{equation}
\D = \{U \cap A \mid U \text{ on pisteen }x\text{ ympäristö}\}\text{.}
\end{equation}

Koska piste $x$ kuuluu joukon $A$ sulkeumaan, jokainen pisteen $x$ ympäristö leikkaa joukon A, eli jokainen joukon $\D$ alkio on epätyhjä joukko. Voimme siis liittää jokaiseen pisteen $x$ ympäristöön $U$ jonkin pisteen $a_U \in A \cap U$. Voimme siis rakentaa kuvauksen $n : \D \to A$, $n(U) \in A \cap U$. Emme määrittele funktiota tuon tarkemmin, meille riittää tietää että jokaiselle alkiolle voidaan tämän ehdon toteuttava funktion arvo määrätä.

Nyt voimme suunnata joukon $\D$ relaatiolla $\supset$. Relaatio $\supset$ tunnetusti toteuttaa transitiivisuuden ja refleksiivisuyyden, eli ehdot V1 ja V2. Ehden V3 toteutumiseksi huomaamme, että mikäli valitsemme kaksi mielivaltaista avaruuden $\D$ jäsentä $D_1, D_2 \in \D$, tällöin $D_n$ on muotoa $A \cap U_n$, jossa $U_n$ on pisteen $x$ ympäristö. Topologian määritelmän ehdon T2 mukaan siis leikkaus $U_1 \cap U_2 =: U$ on myös avoin joukko. Toisaalta koska joukko $U$ sisältää pisteen $x$, on $U$ myös pisteen $x$ ympäristö, eli $A \cap U$ sisältyy joukkoon $\D$. Toisaalta $D_1 = U_1 \cap A \supset U \cap A$ ja $D_2 = U_2 \cap A \supset U \cap A$. Täten joukon $\D$ alkio $U \cap A$ toteuttaa ehtomme.

Funktio $n : \D \to A$ on siis verkko. Jäljellä on enää osoittaa että $n$ suppenee kohti pistettä $x$.

Olkoon $U$ jokin pisteen $x$ ympäristö. Meidän on löydettävä joukon $\D$ alkio $d_U$ jolle pätee

\begin{equation}
\text{jos } d_U \supset d \text{, tällöin } n(d) \in U\text{.}
\end{equation}

Tällainen alkio löytyy muodossa $d_U = A \cap U$. Jos jokin joukon $\D$ alkio $d$ sisältyy alkioon $d_U$, eli pätee $d_U \supset d$, tällöin piste $d$ on muotoa $d = V \cap A$, joten $V \subset U$. Täten $n(d) \in V \subset U$. Täten verkko $n$ suppenee kohti pistettä $x$.

\begin{thebibliography}{9}

\bibitem{Topo}
Jussi Väisälä: Topologia II. Limes ry. 2005
\bibitem{Topo1}
Jussi Väisälä: Topologia I. Limes ry. 2007
\bibitem{Kelley}
J. L. Kelley: General Topology. D. Van Nostrand Company Inc. 1955.
\end{thebibliography}

\end{document}
